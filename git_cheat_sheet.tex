\documentclass[12pt]{article}
\usepackage{amsmath}
\usepackage{graphicx}
\usepackage{hyperref}
\usepackage{fullpage}
\usepackage{fancyhdr}
\usepackage{url}
\usepackage{color}
\usepackage{textcomp}
\usepackage{geometry}
\usepackage{courier}
\geometry{
  top=1.0in,            % <-- you want to adjust this
  inner=0.5in,
  outer=0.5in,
  bottom=1.0in,
  headheight=4ex,       % <-- and this
  headsep=3ex,          % <-- and this
}

\renewcommand{\familydefault}{\sfdefault}

\addtolength{\parskip}{\baselineskip}  
\pdfpagewidth 8.5in
\pdfpageheight 11in
\pagestyle{fancy}
\newcommand{\urlwofont}[1]{\urlstyle{same}\url{#1}}

\renewcommand{\headrulewidth}{0pt}
\renewcommand{\footrulewidth}{0pt}
\lhead{\leftmark}
\chead{}
\rhead{\rightmark}
\lfoot{}
\cfoot{Page \thepage}
\rfoot{}
\hypersetup{
    colorlinks=false,
    linkcolor=blue,
    citecolor=black,
    filecolor=black,
    urlcolor=blue
}
\begin{document}
\section{git Introduction}
A Version Control System (VCS) manages the entire edit history of your code base (or ``repository''. It's helpful in a lot of ways: to go back to previous versions of your code, to revert bad changes, to make collaborating easier, to name a few.

git is a distributed VCS. git is nice because of its speed and its ability to efficiently handle merges and manage branches. We hope you share our opinions regarding this :)

\section{git commands}

\begin{enumerate}

\item{\texttt{help}}  
     \subitem{Show help for a git command.}
     \subitem{Example: \texttt{git help}; \texttt{git help push}; \texttt{git help help}}

\item{\texttt{init}}  
     \subitem{Create a new repo in the current folder.}
     \subitem{Example: \texttt{git init}}

\item{\texttt{clone}}
     \subitem{Copies the repository given, including all the commits and branches.}
     \subitem{Example: \texttt{git clone git@github.com:UP-MIT-GSL/angular-momentum.git}}

\item{\texttt{add}}
     \subitem{Add (``stage'') the changes for the next commit.}
     \subitem{Example: \texttt{git add edited/file.html}; \texttt{git add edited/folder/}}

\item{\texttt{status}}
     \subitem{Display status of your working tree (which files have changed, appeared, disappeared, etc.)}
     \subitem{Example: \texttt{git status}}

\item{\texttt{commit}}
     \subitem{Saves the changes (commit) made on the added files to the current branch.}
     \subitem{Usage: \texttt{-m} for adding a message to the commit, \texttt{-a} for adding all the modified files to the changes.}
     \subitem{Example: \texttt{git add index.html; git commit -m "added index.html"}}

\item{\texttt{merge}}
     \subitem{Combines your current branch with the branch to be merged with.}
     \subitem{This command is intelligent for it will automatically combine the two if they have no conflicts.}
     \subitem{e.g. in branch 1 line 256 of file A was changed, while in branch 2 line 512 of file A was changed. \texttt{git merge} detects no conflicts and merges the changes correctly.}
     \subitem{If there is a conflict, it will show which files and lines it occurred. You have to manually resolve the merges and finish the merge commit.}
     \subitem{Usage: \texttt{git merge <branch\_to\_be\_merged\_with>}}
     \subitem{Example: \texttt{git checkout master; git merge fix-website-title}}

\item{\texttt{fetch}}
     \subitem{Get all commits/branches from a remote repository (see ``remote branches'' below).}
     \subitem{Doesn't merge it yet with your branches.}
     \subitem{Usage: \texttt{--all} fetch from all remotes you are tracking.}
     \subitem{Example: \texttt{git fetch origin}; \texttt{git fetch --all}}

\item{\texttt{pull}}
     \subitem{A \texttt{fetch} followed by a \texttt{merge} (essentially).}
     \subitem{Usage: \texttt{git pull <repository> <branch>}}
     \subitem{Example: \texttt{git pull}; \texttt{git pull upstream master}}

\item{\texttt{rm}}
     \subitem{Remove files in the next commit.}
     \subitem{Usage: \texttt{-r} recursive}
     \subitem{Example: \texttt{git rm file/to/be/removed.html}; \texttt{git rm folder/to/be/removed/}}

\item{\texttt{push}}
     \subitem{Push all commits in the current branch to a remote branch (see ``remotes'' below.)}
     \subitem{Usage: \texttt{git push <repository> <branch>}}
     \subitem{Example: \texttt{git push}; \texttt{git push origin master}}

\item{\texttt{branch}}
     \subitem{Display all branches, create/delete a branch.}
     \subitem{Usage: \texttt{git branch} displays all branches, \texttt{git branch <branch\_name>} creates a new branch, \texttt{git branch -d <branch\_name>} deletes a branch (use \texttt{-D} to force delete)}
     \subitem{Example: \texttt{git branch}; \texttt{git branch new-branch}}

\item{\texttt{checkout}}
     \subitem{1. Go to a branch}
     \subitem{Usage: \texttt{git checkout <branch>}, \texttt{-b} to create new branch at the same time}
     \subitem{Example: \texttt{git checkout my-branch}; \texttt{git checkout -b new-branch}}
     \subitem{2. Undo changes to a file (sort of)}
     \subitem{Usage: \texttt{git checkout <file/folder>}}
     \subitem{Example: \path{git checkout file/with/bad/edits.html}; \path{git checkout folder/with/bad/edits/}}

\item{\texttt{log}}
     \subitem{Display the last few commits in your branch/repository.}
     \subitem{Usage: \texttt{--all} to display all branches, \texttt{--graph} to display a (directed acyclic) graph of your commits, \texttt{oneline} to view each commit log in one line}
     \subitem{Example: \texttt{git log}; \texttt{git log --graph --oneline}}

\item{\texttt{diff}}
     \subitem{Compare two branches.}
     \subitem{Example: \texttt{git diff branch-1 branch-2}; \texttt{git diff head\textasciicircum1 head}}

\item{\texttt{remote}}
     \subitem{View remotes, add/modify/delete a remote (see ``remotes'' below).}
     \subitem{Usage:}
        \subsubitem{\texttt{git remote} - view remotes}
        \subsubitem{\texttt{git remote -v} - view remotes (verbose)}
        \subsubitem{\texttt{git remote add <remote-name> <url>} - add a new remote}
        \subsubitem{\texttt{git remote set-url <remote-name> <url>} - modify remote}
        \subsubitem{\texttt{git remote rename <remote-name> <new-name>} - rename remote}
        \subsubitem{\texttt{git remote rm <remote-name>} - remove remote}
     \subitem{Example: \texttt{git remote -v};}
     \subitem{\path{git remote add upstream git@github.com:UP-MIT-GSL/angular-momentum.git}}

\item{\texttt{stash}}
     \subitem{Save your current changes temporarily (in a Last-in-first-out stack).}
     \subitem{Useful if you want to switch to a branch temporarily, for example to read a file from a different branch.}
     \subitem{Usage: \texttt{git stash} to push current changes to the stack, \texttt{git stash pop} to revert saved changes}

\item{\texttt{reset}}
     \subitem{Reset changes and move to a specified branch.}
     \subitem{Usage: \texttt{--soft} move current branch to specified commit, but current changes are retained; \texttt{--hard} move current branch to specified commit and lose current changes.}
     \subitem{If commit is not specified, defaults to \texttt{HEAD}.}
     \subitem{Example: \texttt{git reset --hard} (resets all current changes) \texttt{git reset --soft a8baa8c9}}

\end{enumerate}

\section{git 101}
\subsection{Merge conflicts}
Probably the most important skill to master while using a version-control system (VCS) is to learn how to resolve merge conflicts properly.

TODO teach how to do it properly

\subsection{Stash}
TODO
Please explain concepts (like branching, remotes, stash, etc.) clearly.

\subsection{Branches}
One of the more powerful features of git is its ability to do branching well. [TODO]

TODO explain HEAD

\subsection{Remotes}
TODO
Please explain concepts (like branching, remotes, stash, etc.) clearly.

\subsection{.gitignore}
TODO

TODO also explain \texttt{.gitkeep}, and that it's merely a convention.


\subsection{ssh keys}
TODO
Please explain concepts (like branching, remotes, stash, etc.) clearly.

\subsection{github}
TODO explain github essentials

\subsubsection{forking a repo}
TODO
\subsubsection{making a pull request}
TODO

\subsection{Config files}
Within your repository, there's a file called \texttt{.git/config} which contains [TODO description of .git/config file] (\texttt{.git} is a directory which contains all the data git uses to manage the repository).

Global files are located in \texttt{\urlwofont{~}/.gitconfig}, which is your ``global'' config file (affects all repositories).

TODO explain how to edit config file using \texttt{git config}.

\section{Advanced git}

\subsection{More commands}
If you are brave enough to go to this section, you must realize that \texttt{git help <command>} will give more details about the command.

\begin{enumerate}

\item{\texttt{config}}
     \subitem{TODO}
     \subitem{Example: \texttt{TODO}}

\item{\texttt{cherry-pick}}
     \subitem{TODO}
     \subitem{Example: \texttt{TODO}}

\item{\texttt{rebase}}
     \subitem{TODO}
     \subitem{Example: \texttt{TODO}}

\item{\texttt{bisect}}
     \subitem{Find by binary search the change that introduced a bug}
     \subitem{Example: \texttt{TODO}}

\item{\texttt{grep}}
     \subitem{TODO}
     \subitem{Example: \texttt{TODO}}

\item{\texttt{show}}
     \subitem{TODO}
     \subitem{Example: \texttt{TODO}}

\item{\texttt{tag}}
     \subitem{TODO}
     \subitem{Example: \texttt{TODO}}

\item{\texttt{mv}}
     \subitem{TODO}
     \subitem{Example: \texttt{TODO}}

\end{enumerate}

\subsection{Aliases}
A git alias is a way to avoid typing long common commands over and over again. For example, if you find yourself typing the command \texttt{git log --graph --oneline} over and over again, you can make an alias by editing your \texttt{.gitconfig} file and adding:

\texttt{[alias]\\
graph = log --graph --online
}

You can now use \texttt{git graph}, which is now identical to \texttt{git log --graph --oneline} !

You can use this for abbreviation, e.g. \texttt{git cmt} instead of \texttt{git commit}, etc. You can still pass flags/arguments.

You can also use the underlying shell!

\texttt{[alias]\\
hi = !echo "Hello world!"
}

So \texttt{git hi} will print ``Hello world!''


\subsubsection{Useful aliases}
Some aliases that are very nice (that I got from the internet)

\texttt{[alias]\\
lol = log --graph --decorate --oneline --abbrev-commit\\
lola = log --graph --decorate --oneline --abbrev-commit --all\\
hey = log --graph --format=format:'%C(bold blue)%h%C(reset) - %C(bold green)(%ar)%C(reset) %C(white)%s%C(reset) %C(bold white)— %an%C(reset)%C(bold yellow)%d%C(reset)' --abbrev-commit --date=relative\\
heya = log --graph --all --format=format:'%C(bold blue)%h%C(reset) - %C(bold green)(%ar)%C(reset) %C(white)%s%C(reset) %C(bold white)— %an%C(reset)%C(bold yellow)%d%C(reset)' --abbrev-commit --date=relative 
}

You can now use \texttt{git lol} or \texttt{git lola} or \texttt{git heya} for very nice graphs of your commits!

\subsubsection{Other aliases}
Visit \url{https://git.wiki.kernel.org/index.php/Aliases} for more cool aliases.
\subsection{Useful \texttt{.gitconfig} options}

\subsection{text editor}
If you're not yet familiar with the text editor \texttt{vim} and would like to use another one for writing commit messages, add the following to your \texttt{.gitconfig} file (to use the \texttt{nano} text editor).

\texttt{[core]\\
editor = nano }

Note, press \texttt{Ctrl+X} to exit \texttt{nano}

\subsection{mergeconflict style}
If, while resolving merge conflicts, you'd like to see the original version, in addition to your edited version and their edited version, add the following to your \texttt{.gitconfig} file.

\texttt{[merge]\\
conflictstyle = diff3 }

\subsection{coloring}
Add the following to your \texttt{.gitconfig} file to add coloring to git outputs like \texttt{git log} or \texttt{git status}.

\texttt{[color]\\
  branch = auto\\
  diff = auto\\
  interactive = auto\\
  status = auto }

\section{Online interactive lesson}
If you want to learn interactively, go to \url{http://pcottle.github.io/learnGitBranching/} . Have fun :)

\end{document} 

​
