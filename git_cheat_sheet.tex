\documentclass[12pt]{article}
\usepackage{amsmath}
\usepackage{graphicx}
\usepackage{hyperref}
\usepackage{fullpage}
\usepackage{fancyhdr}
\usepackage{url}
\usepackage{color}
\usepackage{textcomp}
\usepackage{geometry}
\usepackage{courier}
\geometry{
  top=1.0in,            % <-- you want to adjust this
  inner=0.5in,
  outer=0.5in,
  bottom=1.0in,
  headheight=4ex,       % <-- and this
  headsep=3ex,          % <-- and this
}

\renewcommand{\familydefault}{\sfdefault}

\addtolength{\parskip}{\baselineskip}  
\pdfpagewidth 8.5in
\pdfpageheight 11in
\pagestyle{fancy}
\newcommand{\urlwofont}[1]{\urlstyle{same}\url{#1}}

\renewcommand{\headrulewidth}{0pt}
\renewcommand{\footrulewidth}{0pt}
\lhead{\leftmark}
\chead{}
\rhead{\rightmark}
\lfoot{}
\cfoot{Page \thepage}
\rfoot{}
\hypersetup{
    colorlinks=false,
    linkcolor=blue,
    citecolor=black,
    filecolor=black,
    urlcolor=blue
}
\begin{document}
\section{git Introduction}
TODO Explain (BRIEFLY!) what a VCS is, what git is, and why it's awesome.

TODO Explain what a repository is.

\section{git commands}

\begin{enumerate}

\item{\texttt{help}}  
     \subitem{Show help for a git command.}
     \subitem{Example: \texttt{git help}; \texttt{git help push}; \texttt{git help help}}

\item{\texttt{init}}  
     \subitem{Create a new repo in the current folder.}
     \subitem{Example: \texttt{git init}}

\item{\texttt{clone}}
     \subitem{TODO}
     \subitem{Example: \texttt{TODO}}

\item{\texttt{fetch}}
     \subitem{TODO}
     \subitem{Example: \texttt{TODO}}

\item{\texttt{merge}}
     \subitem{TODO}
     \subitem{Example: \texttt{TODO}}

\item{\texttt{pull}}
     \subitem{TODO}
     \subitem{A \texttt{fetch} followed by a \texttt{merge}.}
     \subitem{Example: \texttt{TODO}}

\item{\texttt{mv}}
     \subitem{TODO}
     \subitem{Example: \texttt{TODO}}

\item{\texttt{rm}}
     \subitem{TODO}
     \subitem{Example: \texttt{TODO}}

\item{\texttt{commit}}
     \subitem{TODO}
     \subitem{Example: \texttt{TODO}}

\item{\texttt{push}}
     \subitem{TODO}
     \subitem{Example: \texttt{TODO}}

\item{\texttt{branch}}
     \subitem{TODO}
     \subitem{Example: \texttt{TODO}}

\item{\texttt{checkout}}
     \subitem{[TODO 1. go to a branch (with optional \texttt{-b}), 2. undo changes to tracked file]}
     \subitem{Example: \texttt{TODO}}

\item{\texttt{remote}}
     \subitem{TODO}
     \subitem{Example: \texttt{TODO}}

\item{\texttt{log}}
     \subitem{TODO}
     \subitem{Example: \texttt{TODO}}

\item{\texttt{status}}
     \subitem{TODO}
     \subitem{Example: \texttt{TODO}}

\item{\texttt{diff}}
     \subitem{TODO}
     \subitem{Example: \texttt{TODO}}

\item{\texttt{stash}}
     \subitem{TODO}
     \subitem{Example: \texttt{TODO}}

\item{\texttt{reset}}
     \subitem{TODO}
     \subitem{Example: \texttt{TODO}}

\end{enumerate}

\section{git 101}
\subsection{Merge conflicts}
Probably the most important skill to master while using a version-control system (VCS) is to learn how to resolve merge conflicts properly.

TODO teach how to do it properly

\subsection{Stash}
TODO
Please explain concepts (like branching, remotes, stash, etc.) clearly.

\subsection{Branches}
One of the more powerful features of git is its ability to do branching well. [TODO]

\subsection{Remotes}
TODO
Please explain concepts (like branching, remotes, stash, etc.) clearly.

\subsection{.gitignore}
TODO

TODO also explain \texttt{.gitkeep}, and that it's merely a convention.


\subsection{ssh keys}
TODO
Please explain concepts (like branching, remotes, stash, etc.) clearly.

\subsection{github}
TODO explain github essentials

\subsubsection{forking a repo}
TODO
\subsubsection{making a pull request}
TODO

\subsection{Config files}
Within your repository, there's a file called \texttt{.git/config} which contains [TODO description of .git/config file] (\texttt{.git} is a directory which contains all the data git uses to manage the repository).

Global files are located in \texttt{\urlwofont{~}/.gitconfig}, which is your ``global'' config file (affects all repositories).

TODO explain how to edit config file using \texttt{git config}.

\section{Advanced git}

\subsection{More commands}
\begin{enumerate}

\item{\texttt{config}}
     \subitem{TODO}
     \subitem{Example: \texttt{TODO}}

\item{\texttt{cherry-pick}}
     \subitem{TODO}
     \subitem{Example: \texttt{TODO}}

\item{\texttt{rebase}}
     \subitem{TODO}
     \subitem{Example: \texttt{TODO}}

\item{\texttt{bisect}}
     \subitem{Find by binary search the change that introduced a bug}
     \subitem{Example: \texttt{TODO}}

\item{\texttt{grep}}
     \subitem{TODO}
     \subitem{Example: \texttt{TODO}}

\item{\texttt{show}}
     \subitem{TODO}
     \subitem{Example: \texttt{TODO}}

\item{\texttt{tag}}
     \subitem{TOD}
     \subitem{Example: \texttt{TODO}}

\end{enumerate}

\subsection{Aliases}
TODO

\subsubsection{Useful aliases}
\texttt{[alias]\\
lol = log --graph --decorate --oneline --abbrev-commit\\
lola = log --graph --decorate --oneline --abbrev-commit --all\\
hey = log --graph --format=format:'%C(bold blue)%h%C(reset) - %C(bold green)(%ar)%C(reset) %C(white)%s%C(reset) %C(bold white)— %an%C(reset)%C(bold yellow)%d%C(reset)' --abbrev-commit --date=relative\\
heya = log --graph --all --format=format:'%C(bold blue)%h%C(reset) - %C(bold green)(%ar)%C(reset) %C(white)%s%C(reset) %C(bold white)— %an%C(reset)%C(bold yellow)%d%C(reset)' --abbrev-commit --date=relative 
}

\subsection{Useful \texttt{.gitconfig} options}


\subsection{text editor}

\texttt{[core]\\
editor = nano }

\subsection{mergeconflict style}
\texttt{[merge]\\
conflictstyle = diff3 }

\subsection{coloring}
\texttt{[color]\\
  branch = auto\\
  diff = auto\\
  interactive = auto\\
  status = auto }

\section{Online interactive lesson}
If you want to learn interactively, go to \url{http://pcottle.github.io/learnGitBranching/} . Have fun :)

\end{document} 

​
