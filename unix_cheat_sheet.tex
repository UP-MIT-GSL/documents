\documentclass[12pt]{article}
\usepackage{amsmath}
\usepackage{graphicx}
\usepackage{hyperref}
\usepackage{fullpage}
\usepackage{fancyhdr}
\usepackage{url}
\usepackage{color}
\usepackage{textcomp}
\usepackage{geometry}
\usepackage{courier}
\geometry{
  top=1.0in,            % <-- you want to adjust this
  inner=0.5in,
  outer=0.5in,
  bottom=1.0in,
  headheight=4ex,       % <-- and this
  headsep=3ex,          % <-- and this
}

\renewcommand{\familydefault}{\sfdefault}

\addtolength{\parskip}{\baselineskip}  
\pdfpagewidth 8.5in
\pdfpageheight 11in
\pagestyle{fancy}
\newcommand{\urlwofont}[1]{\urlstyle{same}\url{#1}}

\renewcommand{\headrulewidth}{0pt}
\renewcommand{\footrulewidth}{0pt}
\lhead{\leftmark}
\chead{}
\rhead{\rightmark}
\lfoot{}
\cfoot{Page \thepage}
\rfoot{}
\hypersetup{
    colorlinks=false,
    linkcolor=blue,
    citecolor=black,
    filecolor=black,
    urlcolor=blue
}
\begin{document}

\section{Useful Unix Utilities}

\begin{enumerate}

\item{\texttt{man}}  
     \subitem{Displays manuals (aka \texttt{man} pages). Press \texttt{Q} to exit a \texttt{man} page.}
     \subitem{Type \texttt{/<text>} to search for text. Press \texttt{N} to go to next match, \texttt{shift+N} to go to previous match.}
     \subitem{Usage: \texttt{man ls};\texttt{ man cp};\texttt{ man man}}

\item{\texttt{pwd}}  
     \subitem{Print current directory}
     \subitem{Usage: \texttt{pwd}}

\item{\texttt{clear}} 
     \subitem{Clears the screen (sort of).}
     \subitem{In a Mac, you can also press \texttt{Command+K} at any time to clear the screen, but this differs from \texttt{clear} in that it actually clears the screen.}
     \subitem{Usage: \texttt{clear}}

\item{\texttt{echo}} 
     \subitem{Print a line to the console.}
     \subitem{Usage: \texttt{echo `hello, world'}}

\item{\texttt{ls}}   
     \subitem{List files in a directory (current directory by default)}
     \subitem{Useful Options: \texttt{-r} (recursive), \texttt{-a} (hidden files), \texttt{-l} (verbose)}
     \subitem{Usage: \texttt{ls};\texttt{ ls -al};\texttt{ ls -al \urlwofont{~/}}}

\item{\texttt{cd}}   
     \subitem{Change directory}
     \subitem{Usage: \texttt{cd /};\texttt{ cd ..};\texttt{ cd \urlwofont{~/}}}

\item{\texttt{cp}}   
     \subitem{Copy file (or directory), overwrites destination if it exists}
     \subitem{Useful Options: \texttt{-R}}
     \subitem{Usage: \texttt{cp a.txt b.txt};\texttt{ cp -R folder otherfolder}}

\item{\texttt{mv}}   
     \subitem{Move files}
     \subitem{Warning: overwrites destination if it exists}
     \subitem{Usage: \texttt{mv a.txt b.txt};\texttt{ mv a.txt folder/b.txt};\texttt{ mv a.txt folder}}

\item{\texttt{rm}}   
     \subitem{Delete files}
     \subitem{Useful options: \texttt{-r} (recursive)}
     \subitem{Usage: \texttt{rm a.txt};\texttt{ rm -r folder}}
     \subitem{Trivia: Never run \texttt{rm -rf /}. It will delete everything.}

\item{\texttt{touch}} 
     \subitem{Make an empty file. If file exists, updates last modified date.}
     \subitem{Useful to trigger watchers like guard.}
     \subitem{Useful options: \texttt{-r} (recursive)}
     \subitem{Usage: \texttt{touch -r *}}

\item{\texttt{mkdir}} 
     \subitem{Make an empty directory}
     \subitem{Usage: \texttt{mkdir folder}}

\item{\texttt{rmdir}} 
     \subitem{Delete a directory. It has to be empty}
     \subitem{Usage: \texttt{rmdir folder}}

\item{\texttt{grep}}  
     \subitem{Find all lines in file matching some regular expression.}
     \subitem{Useful options: \texttt{-I} (ignore binary files like images), \texttt{-i} (ignore case), \texttt{-r} (recursive), \texttt{-v} (invert match)}
     \subitem{Usage: \texttt{grep pattern *};\texttt{ grep -iIr nginx puppet/*}}

\item{\texttt{find}}  
     \subitem{Find all files with some property.}
     \subitem{Useful options: \texttt{-name}}
     \subitem{Usage: \texttt{find . -name "*s.coffee"}}

\item{\texttt{curl}}  
     \subitem{Make HTTP requests from console.}
     \subitem{Useful options: \texttt{-d <data>} for form data, \texttt{-X <method>} to specify HTTP method}
     \subitem{Usage:}
       \subsubitem{\texttt{curl www.google.com}}
       \subsubitem{\texttt{curl localhost:8000}}
       \subsubitem{\texttt{curl localhost:8000/api/messages -d "message=hello"}}
       \subsubitem{\texttt{curl localhost:8000/api/messages/1 -d "message=world" -X "PUT"}}

\item{\texttt{ping}} 
     \subitem{Sends data to a remote server. The server will send data back to you if it is up.}
     \subitem{Usage: \texttt{ping www.google.com}; \texttt{ping localhost}}

\item{\texttt{ps}} 
     \subitem{A way to list all processes and their respective process IDs}
     \subitem{Useful options: Use it with \texttt{-ef}}
     \subitem{Note: This is more effective when used with \texttt{grep} since you can now filter/search for a particular process name.}
     \subitem{Usage: \texttt{ps -ef | grep InternetExplorer}}

\item{\texttt{kill}} 
     \subitem{Terminates a process. Expects a process ID.}
     \subitem{Useful options: \texttt{-9} to absolutely terminate a process. The process cannot stop it.}
     \subitem{Note: In order to see the process IDs, use \texttt{ps}.}
     \subitem{Usage: \texttt{kill 1234}; \texttt{kill -9 12345}}

\item{\texttt{chmod}} 
     \subitem{Changes file permissions.}
     \subitem{Use the command as follows: \texttt{chmod <A><B><C>}, where \texttt{<A>}, \texttt{<B>}, and \texttt{<C>} are explained below.}
     \subitem{There are other ways to use the command (see \texttt{man chmod}), but these ones are useful for most purposes.}
     \subitem{Values of \texttt{<A>}:}
        \subsubitem{\texttt{u} (User)}
        \subsubitem{\texttt{g} (Group)}
        \subsubitem{\texttt{o} (Others)}
     \subitem{Values of \texttt{<B>}:}
        \subsubitem{\texttt{+} (Add)}
        \subsubitem{\texttt{-} (Remove)}
     \subitem{Values of \texttt{<C>}:}
        \subsubitem{\texttt{r} (Read permissions)}
        \subsubitem{\texttt{w} (Write permissions)}
        \subsubitem{\texttt{x} (Executable permissions)}
     \subitem{Usage:}
        \subsubitem{\texttt{chmod u+x myscript.sh}}
        \subsubitem{\texttt{chmod o-w myfile.txt}}

\item{\texttt{chown}} 
     \subitem{Change file ownership.}
     \subitem{Use the command as follows: \texttt{chown <owner>:<group> <file>}, \texttt{chown <owner> <file>}, or \texttt{chown :<group> <file>}.}
     \subitem{Useful options: \texttt{-R} (recursive)}
     \subitem{Usage: \texttt{chown alvin:up-mit-gsl myfile.txt}}

\item{\texttt{sudo}}  
     \subitem{Do action as another user (\texttt{root} by default)}
     \subitem{Useful options: \texttt{-u <username>}}
     \subitem{Usage: \texttt{sudo touch a.txt};\texttt{ sudo rm a.txt}}

\item{\texttt{nano}}  
     \subitem{Simple console-based text editor}
     \subitem{Usage: \texttt{nano a.txt}}

\item{\texttt{tail}}  
     \subitem{Display last few lines of a file. Useful for watching log files}
     \subitem{Useful options: \texttt{-f} (wait for more data)}
     \subitem{Usage: \texttt{tail a.txt};\texttt{ tail -f /var/log/upstart/guard.log}}

\item{\texttt{head}}  
     \subitem{Display top few lines of a file. Not as useful as \texttt{tail}.}

\item{\texttt{cat}}   
     \subitem{Print out contents of a file to the console}
     \subitem{Useful options: \texttt{-n} (enumerate lines), \texttt{-s} (compress blank lines)}
     \subitem{Usage: \texttt{cat a.txt};\texttt{ cat -sn a.txt}}

\item{\texttt{less}}  
     \subitem{View the contents of a file. Press \texttt{Q} to exit.}
     \subitem{Commands similar to \texttt{man}.}
     \subitem{Usage: \texttt{less a.txt};\texttt{ cat -n a.txt | less}}

\item{\texttt{diff}}  
     \subitem{Compare two files}
     \subitem{Useful options: \texttt{-i} (ignore case), \texttt{-b} (ignore blank lines)}
     \subitem{Usage: \texttt{diff a.txt b.txt};\texttt{ diff {a,b}.txt}}

\item{\texttt{wc}}    
     \subitem{Word count tool; useful for counting lines of code and things.}
     \subitem{Useful options: \texttt{-c} (characters), \texttt{-w} (words), \texttt{-l} (lines)}
     \subitem{Usage: \texttt{wc -l a.txt}}

\item{\texttt{ssh}}   
     \subitem{Log into a remote computer or virtual machine}
     \subitem{Usage: \texttt{ssh username@hostname};\texttt{ ssh payton@kalibrr.com}}

\item{\texttt{scp}}   
     \subitem{It's like \texttt{cp}, but between multiple computers.}
     \subitem{Usage: \texttt{scp a.txt payton@kalibrr.com:/home/payton/a.txt}}

\item{\texttt{apt-get}} 
     \subitem{Package manager for Ubuntu (and other Debian-based distros)}
     \subitem{Installs and removes things.}
     \subitem{Useful commands:}
          \subsubitem{\texttt{install} - Installs a package}
          \subsubitem{\texttt{remove}  - Removes a package}
          \subsubitem{\texttt{purge}   - Removes a package and clears configuration}
          \subsubitem{\texttt{update}  - Updates the cached package listing}
     \subitem{Usage:}
          \subsubitem{\texttt{sudo apt-get install vim}}
          \subsubitem{\texttt{sudo apt-get purge vim}}
          \subsubitem{\texttt{sudo apt-get update}}

\item{\texttt{apt-cache}} 
     \subitem{Tool for searching for packages (Debian-based only)}
     \subitem{Useful commands:}
          \subsubitem{\texttt{search} - search for package with substring}
          \subsubitem{\texttt{show}   - show detailed information about a package}
     \subitem{Usage: \texttt{apt-cache search vi | grep vim};\texttt{ apt-cache show vim}}

\item{\texttt{initctl}}   
     \subitem{Tool for managing services (Ubuntu only)}
     \subitem{Useful commands:}
           \subsubitem{\texttt{start} - starts a service}
           \subsubitem{\texttt{stop}  - stops a service}
     \subitem{Usage: \texttt{sudo initctl start guard};\texttt{ sudo initctl stop nginx}}

\item{\texttt{service}}   
     \subitem{Another tool for managing services (Ubuntu only)}
     \subitem{Usage: \texttt{sudo service guard start};\texttt{ sudo service nginx stop}}

\item{\texttt{open}} 
     \subitem{Opens a file with default program (Mac only)}
     \subitem{Usage: \texttt{open a.png} (opens with Preview)}

\end{enumerate}

\newpage
\section{Unix Command Line 101}
\subsection{Directory Naming}

If it starts with \texttt{/}, it's an absolute path that starts from the root directory ( \texttt{/} ).

If it starts with \texttt{\urlwofont{~/}}, it's an absolute path that starts from the user's home directory (\texttt{/Users/username/} on Mac; \texttt{/home/username/} on Linux).

Everything else is a path relative to the current directory (see \texttt{pwd}).

`\texttt{..}' means one folder up.

`\texttt{.}' means current folder.

\subsubsection{Examples}
\begin{enumerate}
\item \texttt{cd ..} brings you one folder up.
\item \texttt{cd example/..} is the same as \texttt{cd .}
\end{enumerate}

\subsection{Pipes and Redirection}
\begin{enumerate}

\item{\texttt{|}}
  \subitem{Pass the output from a command to another command}
  \subitem{Usage:}
     \subsubitem{\texttt{ls -R | grep ".*\textbackslash.txt"} (Find all files that end with \texttt{.txt})}
     \subsubitem{\texttt{cat a.txt | grep -v "\textasciicircum\$" | less} (Read file without the empty lines)}
  \subitem{This is one of the more powerful features of bash.  Be sure to use it.}

\item{\texttt{<}}
  \subitem{Use a file as input.}
  \subitem{Example: \texttt{java Hello < input.txt} (Run a Java program entitled Hello with \texttt{input.txt} as input.)}

\item{\texttt{>}}
  \subitem{Save output to a file. Overwrites current file.}
  \subitem{Example: \texttt{java Hello < input.txt > output.txt}}

\item{\texttt{>>}}
  \subitem{Append output to a file.}
  \subitem{Example: \texttt{java Hello < input.txt >> output.txt  }}

\item{\texttt{2>}}
  \subitem{Save error output to a file. Overwrites current file.}

\item{\texttt{2>>}}
  \subitem{Append error output to a file.}
\end{enumerate}

\subsection{Globbing}
You can pass a ``glob'' to many commands. A glob is a pattern that's much simpler than regular expressions but is useful nonetheless.

\texttt{*} matches any number of characters including the empty string

\texttt{?} matches exactly one character

\subsubsection{Examples}
\begin{enumerate}
\item \texttt{rm -r *.coffee} (remove all files that end in \texttt{.coffee})
\item \texttt{rm -r ???} (remove files and directories whose names are three characters long)
\end{enumerate}

\subsection{Curly Braces}
Using curly braces (\texttt{\{\}}) in a command causes it to expand the command.

\subsubsection{Examples}
\begin{enumerate}
\item \texttt{diff \{a,b\}.txt} is equivalent to \texttt{diff a.txt b.txt}
\item \texttt{touch \{a,b,c,d\}.\{c,py,java\}} - \texttt{touch}es 12 files
\item \texttt{diff file\{1,2\}.txt} is equivalent to \texttt{diff file1.txt file2.txt}
\end{enumerate}

\newpage
\section{Advanced Stuff}
\subsection{More Unix Commands}
\begin{enumerate}

\item{\texttt{vim}}
  \subitem{Advanced console-based text editor}
  \subitem{Learn on your own. It's very complicated.}

\item{\texttt{emacs}}
  \subitem{Even more complicated and powerful text editor.}
  \subitem{We heard it can even play music.}

\item{\texttt{tar}}
  \subitem{Super duper unzipper. Bahala na kayo sa options.}

\item{\texttt{make}}
  \subitem{Recompiles parts of programs that changed.}
  \subitem{Also complicated. Learn own your own.}
\end{enumerate}

\subsection{More on Bash}
Variables:

To set variables:
\texttt{VARIABLE="hello world"} (No spaces)

To use variables:
\texttt{echo \$VARIABLE}

Note: Variables are case-sensitive.

If you want to initialize some stuff before starting the bash prompt, you can place the codes in \\ \texttt{\urlwofont{~/}.bash\char`_profile}



For more advanced bash scripting, check the Internet.


\end{document} 

​