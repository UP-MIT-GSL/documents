\documentclass[12pt]{article}
\usepackage{amsmath}
\usepackage{graphicx}
\usepackage{hyperref}
\usepackage{fullpage}
\usepackage{fancyhdr}
\usepackage{url}
\usepackage{color}
\usepackage{textcomp}
\usepackage{geometry}
\usepackage{courier}
\geometry{
  top=1.0in,            % <-- you want to adjust this
  inner=0.5in,
  outer=0.5in,
  bottom=1.0in,
  headheight=4ex,       % <-- and this
  headsep=3ex,          % <-- and this
}

\renewcommand{\familydefault}{\sfdefault}

\addtolength{\parskip}{\baselineskip}  
\pdfpagewidth 8.5in
\pdfpageheight 11in
\pagestyle{fancy}
\newcommand{\urlwofont}[1]{\urlstyle{same}\url{#1}}

\renewcommand{\headrulewidth}{0pt}
\renewcommand{\footrulewidth}{0pt}
\lhead{\leftmark}
\chead{}
\rhead{\rightmark}
\lfoot{}
\cfoot{Page \thepage}
\rfoot{}
\hypersetup{
    colorlinks=false,
    linkcolor=blue,
    citecolor=black,
    filecolor=black,
    urlcolor=blue
}
\begin{document}
\newpage
\section{JavaScript and CoffeeScript}
CoffeeScript is a language that compiles to JavaScript.

For more info, see \url{coffeescript.org}.

[TODO Useful CoffeeScript conveniences]

\section{jQuery}
jQuery is an awesome JavaScript library.

For more info, see \url{api.jquery.com} and \url{oscarotero.com/jquery}.

[TODO jQuery stuff]

\section{JSON}
JSON (JavaScript Object Notation) is an awesome text-based human-readable data interchange format. XML is another such format, but it's bloated and ugly.

In Angular Momentum, JSON is the format used when the frontend communicates with the backend, and when the backend responds.

For example, the frontend might issue a request to create a new message like:
\begin{verbatim}
{
    "id": null,
    "message": "Hello world!"
}
\end{verbatim}

And the backend might reply as:
\begin{verbatim}
{
    "ok": true,
    "message": {
        "id": 123,
        "message": "Hello world!",
        "created_at": "2011-11-11 11:11:11"
    }
}
\end{verbatim}



For more info, see \url{json.org}.

\subsection{The JSON format}
A \texttt{JSON value} is one of the following:
\begin{itemize}
\item{\texttt{null}}
    \subitem{Represents nothingness.}

\item{\texttt{Boolean}}
    \subitem{Represents a boolean value (\texttt{true} or \texttt{false}).}

\item{\texttt{Number}}
    \subitem{Represents a number (integer or floating point format).}
    \subitem{Examples: \texttt{11}; \texttt{-123.456}; \texttt{6.02e23}}

\item{\texttt{String}}
    \subitem{Represents a string, delimited by double quotes (`` '').}
    \subitem{Escape quotes (and other special characters) using backslash (\textbackslash).}
    \subitem{Example: \texttt{"Hey\textbackslash nMy name is \textbackslash"John\textbackslash""}}

\item{\texttt{Array}}
    \subitem{Represents an ordered collection of values.}
    \subitem{Each element can be any \texttt{JSON value}.}
    \subitem{The elements are comma-separated (trailing comma not allowed) and can be empty.}
    \subitem{Example: \texttt{[true, 5, "John", [null, "Hey"], []]}}

\item{\texttt{Object}}
    \subitem{Represents an ordered set of name/value pairs.}
    \subitem{Each key must be a \texttt{String}. Each value can be any \texttt{JSON value}.}
    \subitem{Keys must be distinct.}
    \subitem{The key/value pairs are comma-separated (trailing comma not allowed) and can be empty.}
    \subitem{Example: \texttt{\{"a": 1, "b": "c", "hey": [1, 2, 3, \{\}]\}}}

\end{itemize}

\end{document}

​