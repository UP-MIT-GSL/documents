\documentclass[10pt, twocolumn]{article}
\usepackage{amsmath}
\usepackage{graphicx}
\usepackage{hyperref}
\usepackage{fullpage}
\usepackage{fancyhdr}
\usepackage{url}
\usepackage{color}
\usepackage{textcomp}
\usepackage{geometry}
\usepackage{courier}
\usepackage{listings}
\geometry{
  top=1.0in,            % <-- you want to adjust this
  inner=0.5in,
  outer=0.5in,
  bottom=1.0in,
  headheight=4ex,       % <-- and this
  headsep=3ex,          % <-- and this
}

\renewcommand{\familydefault}{\sfdefault}

\addtolength{\parskip}{\baselineskip}  
\pdfpagewidth 8.5in
\pdfpageheight 11in
\pagestyle{fancy}
\newcommand{\urlwofont}[1]{\urlstyle{same}\url{#1}}

\renewcommand{\headrulewidth}{0pt}
\renewcommand{\footrulewidth}{0pt}
\lhead{\leftmark}
\chead{}
\rhead{\rightmark}
\lfoot{}
\cfoot{Page \thepage}
\rfoot{}
\hypersetup{
    colorlinks=false,
    linkcolor=blue,
    citecolor=black,
    filecolor=black,
    urlcolor=blue
}
\begin{document}

\section{Selectors}
Note: assume compatible with IE6+, Firefox, Chrome, Safari and Opera unless specified otherwise

\begin{enumerate}
\item {\texttt{X}}
  \begin{itemize}
    \item Element selector
    \item Targets element \texttt{X} (elements whose name is \texttt{X})
  \end{itemize}
  \begin{lstlisting}[frame=single]
  body {
    color: black;
  }
  input {
    color: white;
  }
  \end{lstlisting}

\item {\texttt{\#X}}
  \begin{itemize}
    \item ID Selector
    \item Selects an element with an ID that is \texttt{X}
  \end{itemize}
  \begin{lstlisting}[frame=single]
  #my-id {
    color: white;
  }
  \end{lstlisting}

\item {\texttt{.X}}
  \begin{itemize}
    \item Class Selector
    \item Selects an element with a class that is \texttt{X}
  \end{itemize}
  \begin{lstlisting}[frame=single]
  .my-class {
    color: blue
  }
  \end{lstlisting}

\item {\texttt{X Y}}
  \begin{itemize}
    \item Descendant Selector
    \item Targets element \texttt{Y} that is a descendant of \texttt{X}
  \end{itemize}
  \begin{lstlisting}[frame=single]
  p a {
    text-decoration: none;
  }
  p .my-class {
    text-decoration: underline;
  }
  \end{lstlisting}

\item {\texttt{X > Y}}
  \begin{itemize}
    \item Child Selector
    \item Targets element \texttt{Y} that is a child (direct descendant) of \texttt{X}
    \item It is recommended to use the child selector instead of the descendant selector for better performance
  \end{itemize}
  \begin{lstlisting}[frame=single]
  ul > li {
    color:  black;
  }
  \end{lstlisting}  

\item {\texttt{X:visited} and \texttt{X:link}}
  \begin{itemize}
    \item \texttt{:visited} targest anchor tags (\texttt{<a>}) that have been clicked
    \item \texttt{:link} targets anchor tags that have not yet been clicked
    \item Compatability: IE7+
  \end{itemize}
  \begin{lstlisting}[frame=single]
  a:visited {
    color: red;
  }
  a:line {
    color: blue;
  }
  \end{lstlisting}

\item {\texttt{X[href="foo"]} and \texttt{X[title]}}
  \begin{itemize}
    \item Selects \texttt{X} with an attribute \texttt{href="foo"}, or with an attribute \texttt{title}.
    \item Compatability: IE7+
  \end{itemize}
  \begin{lstlisting}[frame=single]
  a[href="foo"] {
    color: black;
  }
  #my-id[title] {
    color: gray;
  }
  \end{lstlisting}

\item {\texttt{X:hover}}
  \begin{itemize}
    \item Targets elements that the mouse is over on.
    \item Compatability: IE6+, however in IE6 it is applicable to \texttt{<a>} tags only.
  \end{itemize}
  \begin{lstlisting}[frame=single]
  a:hover {
    text-decoration: underline;
  }
  \end{lstlisting}

\item {\texttt{X:not(Y)}}
  \begin{itemize}
    \item Targets all \texttt{X} except those targeted by \texttt{Y}.
    \item Compatability: IE9+
  \end{itemize}
  \begin{lstlisting}[frame=single]
  button:not(button[disabled]) {
    color: white;
  }
  input:not[.my-class]{
    color: black;
  }
  \end{lstlisting}

\item {\texttt{X:first-child}}
  \begin{itemize}
    \item Targets the first child of the elements parent.  
    \item Compatability: IE7+
  \end{itemize}
  \begin{lstlisting}[frame=single]
  ul li {
    border-top: 1px solid black;
  }
  ul li:first-child {
    border-top: none;
  }
  \end{lstlisting}

\item {\texttt{X:last-child}}
  \begin{itemize}
    \item Similar to \texttt{:first-child} only this time it targets the last child.
    \item Compatability: IE9+
  \end{itemize}

\item \texttt{X[attr\^{}=val]}
\begin{itemize}
\item Element \texttt{X} with an attribute \texttt{attr} value that starts in \texttt{val}
\end{itemize}
\begin{lstlisting}[frame=single]
a[href^=mailto] {
  background: url(emailicon.gif);
}
\end{lstlisting}

\item \texttt{X[attr\$=val]}
\begin{itemize}
\item Element \texttt{X} with an attribute \texttt{attr} value that ends in \texttt{val}
\end{itemize}
\begin{lstlisting}[frame=single]
a[href$=pdf]{
  background: url(pdficon.gif);
}
\end{lstlisting}

\item \texttt{X[attr*=val]}
\begin{itemize}
\item Element \texttt{X} whose attribute \texttt{attr} has a value \texttt{val} that matches anywhere within the attribute 
\end{itemize}
\begin{lstlisting}[frame=single]
[class=*"span"] {
  float: left;
}
\end{lstlisting}

\item \texttt{X:enabled}, \texttt{X:disabled}, \texttt{X:checked} 
\begin{itemize}
\item Style based on the current state of \texttt{X}  
\end{itemize}
\begin{lstlisting}[frame=single]
input[type="checkbox"]:checked {
  color: red;
}
\end{lstlisting}

\item \texttt{X:nth-child(n)} 
\begin{itemize}
\item Target element X that is the nth child of its parent
\item The parameter \texttt{n} accepts an integer
\item If you want to target a variable number of children, you can pass an argument of the form \texttt{an+b}
\item You can also use \texttt{even} and \texttt{odd} as arguments to target even and odd child elements
\item Compatability: IE9+, Firefox 3.5+
\end{itemize}
\begin{lstlisting}[frame=single]
li:nth-child(3) {
  color: red;
}
li:nth-child(2n+1) {
  color: red;
}
li:nth-child(odd) {
  color: red;
}
\end{lstlisting}

\item \texttt{X::first-letter} 
\begin{itemize}
\item Targets the first letter of a body of text
\item Can only be applied to block level elements
\end{itemize}
\begin{lstlisting}[frame=single]
p::first-letter {
  font-weight: bold;
}
\end{lstlisting}

\end{enumerate}

\section{CSS}
\begin{enumerate}

\item{\texttt{color}}  
\begin{itemize}
    \item Changes the text color
    \item Colors can be specified by name (e.g., \path{black}, \path{white}, \path{blue}, \path{red}), by hexadecimal value (e.g, \path{\#fff}, \path{\#000}, \path{\#0000ff}, \path{\#ff0000}) or with the \path{rgb()} function (e.g. \path{rgb(0,0,0)}, \path{rgb(255,255,255)}, \path{rgb(0,0,255)}, \path{rgb(255,0,0)}).
\end{itemize}
\begin{lstlisting}[frame=single]
body {
    color: black;
}
.error {
    color: rgb(255, 0, 0)
}
\end{lstlisting}

\item{\texttt{background-color}}
\begin{itemize}
    \item Changes the background color.  
    \item Takes in the same values as the \texttt{color} property.
\end{itemize}    
\begin{lstlisting}[frame=single]
input {
    background-color: #ecfa01; 
}
\end{lstlisting}
    
\item{\texttt{font-family}}
\begin{itemize}
    \item Sets the font family or ``font face'' of the text.
    \item You can specify multiple font families by separating them with commas.  The browser will apply the fonts in order of first appearance from left to right, this is good for back-up fonts.
\end{itemize}    
\begin{lstlisting}[frame=single]
body {
  font-family: "Times New Roman"
}
p {
  font-family: "Times New Roman", serif
}
\end{lstlisting}
    

\item{\texttt{font-size}}
\begin{itemize}
    \item Sets the size of the text.  
    \item The common units are pixels (\texttt{px}), ems (\texttt{em}), and percentages (\texttt{\%}).  
    \item You can also use the following keywords (\texttt{xx-small}, \texttt{x-small}, \texttt{small}, \texttt{medium}, \texttt{large}, \texttt{x-large}, \texttt{xx-large})
\end{itemize}    
\begin{lstlisting}[frame=single]
body {
  font-size: 12px;
}
p {
  font-size: 120%;
}
input {
  font-size: 1em;
}
button {
  font-size: xx-large;
}
\end{lstlisting}
    
\item{\texttt{line-height}}
\begin{itemize}
    \item Specifies the height of the line of text.  
    \item This is different from \texttt{font-size}.  A larger value for \texttt{line-height} than \texttt{font-size} will increase the space above and below the text.  Similary, a smaller \texttt{line-height} than \texttt{font-size} will cause words in different lines to overlap.  
    \item This property can accept the same values that \texttt{font-size} accepts.
\end{itemize}    
\begin{lstlisting}[frame=single]
body {
  line-height: 10px;
}

\end{lstlisting}
    
\item{\texttt{text-align}}
\begin{itemize}
    \item Aligns the text according to the values \texttt{left}, \texttt{right}, \texttt{center} and \texttt{justify}.
\end{itemize}    
\begin{lstlisting}[frame=single]
.quote {
  text-align: right;
}
\end{lstlisting}
    
\item{\texttt{text-decoration}}
\begin{itemize}
    \item Decorate the text with an  \texttt{underline} ,  \texttt{overline}, or  \texttt{line-through}.
\end{itemize}    
\begin{lstlisting}[frame=single]
.link {
  text-decoraton: underline
}
\end{lstlisting}
    

\item{\texttt{width} and \texttt{height}}
\begin{itemize}
    \item Used to resize block elements, commonly specified with the \texttt{px} and \texttt{\%} units.  
    \item If you use \texttt{\%}, note that \texttt{100\%} means it will cover 100\% of the parent, \texttt{50\%} means it will cover half of the parent, and so on.  
    \item Values with the \texttt{px} unit are fixed as usual.
\end{itemize}    
\begin{lstlisting}[frame=single]
.avatar {
  width: 100%;
  height: 100px;
}
\end{lstlisting}
    
\item{\texttt{margin}}
\begin{itemize}
    \item Specify a space around the element starting from the outer borders of the element, commonly used with \texttt{px} and \texttt{\%}.  
    \item You can use \texttt{margin-top}, \texttt{margin-right}, \texttt{margin-bottom} and \texttt{margin-left} to target specific areas, otherwise, margin will apply to all four regions.
    \item If you set margin to auto the browser will make the element take as much space as possible.
\end{itemize}    
\begin{lstlisting}[frame=single]
input {
  margin: 10px; 
}
p {
  margin-left: 10px;
  margin-right: 23px;
  margin-bottom: auto;
}
\end{lstlisting}
    
\item{\texttt{padding}}
\begin{itemize}
    \item Specify an area inside the element starting from the borders.
    \item Accepts values that \texttt{margin} accepts.
\end{itemize}    
\begin{lstlisting}[frame=single]
input {
  padding: 4px;
}
p {
  padding-top: 20px;
}
\end{lstlisting}
\item{\texttt{z-index}}
\begin{itemize}
    \item Specifies the stack order of positioned elements.  
    \item Elements with a higher \texttt{z-index} value will appear on top.  
    \item Negative numbers are accepted.  
    \item The default \texttt{z-index} is 1.
\end{itemize}    
\begin{lstlisting}[frame=single]
.warning {
  position: relative;
  z-index: 2;
}
\end{lstlisting}

\item{\texttt{float}}
\begin{itemize}
    \item Removes the element from the flow and positions the element either \texttt{left} or \texttt{right}. 
    \item The rest of the content will flow around the floated element.
\end{itemize}    
\begin{lstlisting}[frame=single]
img {
  float: left;
}
.caption {
  float: right;
}
\end{lstlisting}

\item{\texttt{clear}}
\begin{itemize}
    \item Speciying this will tell the browser not to allow any floated elements to the left or right of the cleared element.
\end{itemize}    
\begin{lstlisting}[frame=single]
p {
  clear: left
}   
\end{lstlisting}
\end{enumerate}

\section{Twitter Bootstrap}
\begin{enumerate}
\item Enable Responsiveness
\begin{itemize}
\item Include \texttt{bootstrap-responsive.css} in your main Jade file (\texttt{index.jade}) after \texttt{bootstrap.css}
\end{itemize}  

\item Grid System
\begin{itemize}
\item The Bootstrap grid system consists of 12 columns that make up a fixed 940px-wide container.
\item With \texttt{bootstrap-responsive.css} included, the grid adapts to 724px to 1170px.  Below 767px, your columns will stack vertically.
\item Important classes: \texttt{.row}, \texttt{.span*}, \texttt{.row-fluid}
\item The \texttt{.row class} specifies the row of your grid.  It will take up the whole width of the parent.
\item Under \texttt{.row}, put the appropriate \texttt{.span*} classes.  Remember I said we have 12 columns?  The span classes are \texttt{.span1}, .\texttt{span2}, \texttt{.span3},..., \texttt{.span12}.  The bigger the number, the wider the \texttt{.span*}.
\item Make sure that your \texttt{.span*} classes are under a \texttt{.row}.
\item For example, a two-column layout in Jade:
\begin{lstlisting}[frame=single]
.row
  .span6
    // content for first column
  .span6
    // content for second column
\end{lstlisting}
\item A 3-column layout with unequal widths
\begin{lstlisting}[frame=single]
.row
  .span2
    // content
  .span8
    // main content so it's bigger
  .span2
    // more content
\end{lstlisting}
\item Also you can nest columns
\begin{lstlisting}[frame=single]
.row
  .span9
    //Level 1 column
  .row
    .span6
      //content
    .span3
      //content
\end{lstlisting}
\item Stacking columns?
\begin{lstlisting}[frame=single]
.row
  .span4
  .span4
  .span4
.row
  .span3
  .span3
  .span2
  .span1
  .span6 
\end{lstlisting}
\item \texttt{.row-fluid} is similar to \texttt{.row} only it isn't fixed (uses \texttt{\%} instead of pixels).  It automatically adjusts to the size of the parent.
\end{itemize}

\item Layout
\begin{enumerate}
\item \texttt{.unstyled}
\begin{itemize}
\item Your browser renders bullet points for each item in your unordered lists (\texttt{ul}) by default.  If you don't want any bullets, add \texttt{.unstyled} to your \texttt{ul} tag
\begin{lstlisting}[frame=single]
ul.ustyled
  li Bacon 
  li Tuna
  li Pitchfork
  li Avocado
\end{lstlisting}
\end{itemize}
\end{enumerate}

\item Tables
\begin{enumerate}
\item \texttt{.table}
\begin{itemize}
\item Use this class to create nicely spaced columns with horizontal dividers.
\end{itemize}
\item \texttt{.table-striped}
\begin{itemize}
\item Makes the table rows alternately gray and white for better readability.
\end{itemize}
\end{enumerate}

\item Buttons
\begin{enumerate}
\item \texttt{.btn}
\begin{itemize}
\item Gives the button rounded corners and a gradient.
\item When using any of the non-basic button classes, include \texttt{.btn} as well.  This is because \texttt{.btn} defines the border and hover affects and the rest only change the colors and sizes.  So to get the expected Bootstrap style do something like this:
\begin{lstlisting}[frame=single]
button.btn.btn-primary Click me!
\end{lstlisting}
\end{itemize}

\item \texttt{.btn-primary}
\begin{itemize}
\item The name speaks for itself.  Put this on important buttons in your page.  Bootstrap colors it blue by default but you may want you can change it to match your brand color (Approach a TA if you want to do this and can't figure it out).
\end{itemize}

\item \texttt{.btn-info}
\begin{itemize}
\item The go-to button for displaying information.
\item Light blue by default.
\end{itemize}

\item \texttt{.btn-warning}
\begin{itemize}
\item Yellow by default.
\end{itemize}

\item \texttt{.btn-success}
\begin{itemize}
\item Green by default.
\end{itemize}

\item \texttt{.btn-danger}
\begin{itemize}
\item Red by default.
\item Good visual cue for destructive actions (like deleting something permamently).
\end{itemize}

\item \texttt{.btn-inverse}
\begin{itemize}
\item Just a very dark gray button with no semantic meaning.
\end{itemize}

\item \texttt{.btn-large}, \texttt{.btn-small}, \texttt{.btn-mini}
\begin{itemize}
\item Sizes the buttons
\end{itemize}
\begin{lstlisting}[frame=single]
button.btn.btn-large I'm a big button
button.btn.btn-danger.btn-mini Clear
\end{lstlisting}


\end{enumerate}

\end{enumerate}

\section{CSS3 Properties}
\begin{enumerate}

\item \texttt{border-radius}
\begin{itemize}
\item Makes rounded corners
\item Commonly used with \texttt{px} and \texttt{\%}
\end{itemize}
\begin{lstlisting}[frame=single]
div {
  -webkit-border-radius: 3px;
  -moz-border-radius: 3px;
  -ms-border-radius: 3px;
  -o-border-radius: 3px;
  border-radius: 3px;
}
\end{lstlisting}

\item \texttt{opacity}
\begin{itemize}
\item Adjusts the transparency of an element 
\item Accepts a unitless number from 0 to 1, 0 makes the element completely transparent and 1 makes it completey opaque
\end{itemize}
\begin{lstlisting}[frame=single]
div {
  opacity: 0.5;
}
\end{lstlisting}

\item \texttt{rgba(R, G, B, opacity)}
\begin{itemize}
\item A function for setting a color with transparency
\item Used with \texttt{color} and \text{bacgkround-color}
\end{itemize}
\begin{lstlisting}[frame=single]
div {
  color: rgba(10, 10, 15, 0.75);
}
\end{lstlisting}

\item \texttt{linear-gradient}
\begin{itemize}
\item A function that creates an image with a linear gradient
\item Applied to \texttt{background-image}
\item Gradients are tricky and accept a variable number of parameters, I will give an example of a top to bottom gradient below. If you wish to create more complicated images, approach Cassie the TA.
\item You can make lots of complex patterns with gradients, check out http://lea.verou.me/css3patterns/ for awesome examples.
\item Note: If your gradient rules become too long and increase the file size of your stylesheet, might as well use a separate SVG or PNG file.
\item The browser support of the gradient function depends on which syntax you pick, let's just say you can support IE6+ if you use all of the vendor prefixes below. (Is it worth it?)
\end{itemize}
\begin{lstlisting}[frame=single]
div {
  /* FF 3.6+ */  
  background-image: 
    -moz-linear-gradient(black, white); 
  /* Safari 4+, Chrome 2+ */  
  background-image: 
    -webkit-gradient(linear, 
    left top, left bottom, 
    color-stop(0%, #000000), 
    color-stop(100%, #ffffff));
  /* Safari 5.1+, Chrome 10+ */   
  background-image: 
    -webkit-linear-gradient(black, white); 
  /* Opera 11.10 */  
  background-image: 
    -o-linear-gradient(black, white);
  /* IE6 & IE7 */  
  filter: progid:DXImageTransform.
    Microsoft.gradient(startColorstr=
    '#000000', endColorstr='#ffffff');  
  /* IE8+ */ 
  -ms-filter: "progid:DXImageTransform.
    Microsoft.gradient(startColorstr=
    '#000000', endColorstr='#ffffff')";  
  background-image: 
    linear-gradient(black, white);
}
\end{lstlisting}

\item \texttt{radial-gradient}
\begin{itemize}
\item A function that creates an image with a radial gradient
\item Applied to \texttt{background-image}
\item Compatability: IE10, Firefox 3.6+
\end{itemize}
\begin{lstlisting}[frame=single]
div {
  /* Safari 4-5, Chrome 1-9 */
  background-image: 
    -webkit-gradient(radial, 
    center center, 0,
    center center, 460, 
    from(#1a82f7), to(#2F2727)); 
  /* Safari 5.1+, Chrome 10+ */
  background-image: 
    -webkit-radial-gradient(circle, 
    #1a82f7, #2F2727);
  /* Firefox 3.6+ */ 
  background-image: 
    -moz-radial-gradient(circle, 
    #1a82f7, #2F2727);
  /* IE 10 */ 
  background-image: 
    -ms-radial-gradient(circle, 
    #1a82f7, #2F2727);
}
\end{lstlisting}

\item \texttt{box-shadow}
\begin{itemize}
\item Creates a shadow around or inside your element
\item Syntax:
\begin{lstlisting}[frame=single]
box-shadow: 
    h-shadow v-shadow blur spread color inset;
\end{lstlisting}
\begin{description}
\item[\texttt{h-shadow}] Required.  The horizontal position of the shadow
\item[\texttt{v-shadow}] Required.  The vertical position of the shadow
\item[\texttt{blur}] Optional.  The blur distance
\item[\texttt{spread}] Optional.  The size of the shadow
\item[\texttt{color}] Optional.  The color of the shadow
\item[\texttt{inset}] Optional.  Make the shadow an inner shadow
\end{description}
\item \texttt{box-shadow} accepts a comma separated list of shadow definitions so an element can have more than one shadow
\item Compatability: IE9+
\end{itemize}
\begin{lstlisting}[frame=single]
div {
  box-shadow: 10px 10px 5px #888888;
}
div {
  box-shadow: 10px 10px 5px #888888 inset;
}
div {
  box-shadow: 10px 10px 5px #888888, 
    10px 10px 5px #888888 inset;
}
\end{lstlisting}

\item \texttt{text-shadow}
\begin{itemize}
\item Creates a shadow for your text
\item Syntax:
\begin{lstlisting}[frame=single]
text-shadow: 
    h-shadow v-shadow 
    blur color;
\end{lstlisting}
\begin{description}
\item[\texttt{h-shadow}] Required.  The horizontal position of the shadow
\item[\texttt{v-shadow}] Required.  The vertical position of the shadow
\item[\texttt{blur}] Optional.  The blur distance
\item[\texttt{color}] Optional.  The color of the shadow
\end{description}
\item \texttt{text-shadow} accepts a comma separated list of shadow definitions so an element can have more than one shadow
\item Compatability: IE10
\end{itemize}
\begin{lstlisting}[frame=single]
h1{
  text-shadow: 2px 2px #ff0000;
}
\end{lstlisting}
\item \texttt{transition}
\begin{itemize}
\item Makes an element gradually change styles to give an animated effect
\item Accepts a comma-separate list of property names and duration.
\item The transition takes effect when a property value is changed.  You can use it to make fancier hover effects for example, by changing the color with a smooth transition when the mouse is over an \texttt{a} element.
\item There are a lot more options you can pass for complex effect such as \texttt{linear}, \texttt{cubiz-bezier}, \texttt{ease}, etc.  Look it up online.
\item Compatability: IE10
\end{itemize}
\begin{lstlisting}[frame=single]
div {
  transition: width 2s, height 2s,
    transform 2s;
  -webkit-transition: width 2s, height 2s, 
    -webkit-transform 2s;
}
div {
  transition: color 1s linear 2s;
  -webkit-transition:width 1s linear 2s;
}
\end{lstlisting}
\end{enumerate}

\section{Stylus}
\begin{enumerate}

\item Variables
\begin{itemize}
\item Assign values to variables and use them anywhere within scope
\end{itemize}
\begin{lstlisting}[frame=single]
font-size = 14px

body
 font font-size Arial, sans-serif
\end{lstlisting}

\item Property Lookup
\begin{itemize}
\item Append teh \texttt{@} symbol to a valid property name to access the value
\end{itemize}
\begin{lstlisting}[frame=single]
#logo
   width: 150px
   height: 80px
   margin-left: -(@width / 2)
   margin-top: -(@height / 2)
\end{lstlisting}

\item Mixins
\begin{itemize}
\item A reusable block of code you can insert anywhere in your stylesheet
\item When a mixin is invoked within a selector, the properties are copied into the selector on compilation
\item Use mixins to avoid typing out all the vendor prefixes of CSS3 properties
\end{itemize}
\begin{lstlisting}[frame=single]
border-radius(n)
  -webkit-border-radius n
  -moz-border-radius n
  border-radius n

form input[type=button]
  border-radius(5px)

div
  border-radius(10px)    
\end{lstlisting}

\item Some Built-in Functions
\begin{description}
\item[\texttt{lighten(color, amount)}] Lighten the \texttt{color} by the given \texttt{amount}
\begin{lstlisting}[frame=single]
lighten(#2c2c2c, 30)
// => #787878

lighten(#2c2c2c, 30%)
// => #393939 
\end{lstlisting}

\item[\texttt{darken(color, amount)}] Darken the \texttt{color} by the given \texttt{amount}
\begin{lstlisting}[frame=single]
darken(#D62828, 30)
// => #551010

darken(#D62828, 30%)
// => #961c1c
\end{lstlisting}

\item[\texttt{invert{color}}] Inverts the red, green and blue valus of \texttt{color}, opacity is ignored
\begin{lstlisting}[frame=single]
invert(#d62828)
// => #29d7d7
\end{lstlisting}
\end{description}

\item Operators
\begin{lstlisting}[frame=single]
15px - 5px
// => 10px

"num " + 15
// => "num 15"

2000ms + (1s * 2)
// => 4000ms

5s / 2
// => 2.5s

4 % 2
// => 0

2 ** 8
// => 256

5 == 5
// => true

10 > 5
// => true

#fff == #fff
// => true
\end{lstlisting}

\item Conditionals \texttt{if} / \texttt{else if} / \texttt{else}
\begin{itemize}
\item The \texttt{if} executes the following block if the expression is true.  The \texttt{else if} and \texttt{else} conditions can be used as fallbacks. 
\end{itemize}
\begin{lstlisting}[frame=single]
box(x, y, margin-only)
  if margin-only
    margin y x
  else
    padding y x
\end{lstlisting}

\end{enumerate}


\end{document} 

​



















