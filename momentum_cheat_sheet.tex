\documentclass[12pt]{article}
\usepackage{amsmath}
\usepackage{graphicx}
\usepackage{hyperref}
\usepackage{fullpage}
\usepackage{fancyhdr}
\usepackage{url}
\usepackage{color}
\usepackage{textcomp}
\usepackage{geometry}
\usepackage{courier}
\geometry{
  top=1.0in,            % <-- you want to adjust this
  inner=0.5in,
  outer=0.5in,
  bottom=1.0in,
  headheight=4ex,       % <-- and this
  headsep=3ex,          % <-- and this
}

\renewcommand{\familydefault}{\sfdefault}

\addtolength{\parskip}{\baselineskip}  
\pdfpagewidth 8.5in
\pdfpageheight 11in
\pagestyle{fancy}
\newcommand{\urlwofont}[1]{\urlstyle{same}\url{#1}}

\renewcommand{\headrulewidth}{0pt}
\renewcommand{\footrulewidth}{0pt}
\lhead{\leftmark}
\chead{}
\rhead{\rightmark}
\lfoot{}
\cfoot{Page \thepage}
\rfoot{}
\hypersetup{
    colorlinks=false,
    linkcolor=blue,
    citecolor=black,
    filecolor=black,
    urlcolor=blue
}
\begin{document}
\newpage
\section{Angular Momentum Technology Stack}

Angular Momentum uses technologies like Homebrew, pip, bundle, npm, guard, rake, vagrant.

\section{Homebrew}
Homebrew installs things (in Mac). To install Homebrew, run:

\texttt{ruby -e "\$(curl -fsSL https://raw.github.com/mxcl/homebrew/go)"}

\subsection{Commands}

\begin{itemize}

\item{\texttt{install}}
    \subitem{Install something.}
    \subitem{Usage: \texttt{brew install wget}}

\item{\texttt{uninstall}}
    \subitem{Uninstall something.}
    \subitem{Usage: \texttt{brew uninstall wget}}

\end{itemize}

For a list of things that can be installed/more info, see \url{http://mxcl.github.io/homebrew/}

More tips and tricks can be found in \url{https://github.com/mxcl/homebrew/wiki/Tips-N'-Tricks}

\section{pip}
pip is an awesome Python package manager.

To install pip, [TODO]

For more info, visit \url{http://www.pip-installer.org/en/latest/}

\subsection{Commands}

\begin{itemize}

\item{\texttt{install}}
    \subitem{Install a Python package.}
    \subitem{\texttt{==[version]} to install a specific version.}
    \subitem{\texttt{-r} to install from a requirements file (see ``pip requirements'' below).}
    \subitem{Usage:}
        \subsubitem{\texttt{pip install Flask}}
        \subsubitem{\texttt{pip install Flask==0.9}}
        \subsubitem{\texttt{pip install -r requirements.txt}}

\item{\texttt{uninstall}}
    \subitem{Uninstall a Python package.}
    \subitem{Usage: \texttt{pip uninstall Flask}}

\item{\texttt{list}}
    \subitem{List all installed packages.}
    \subitem{Usage: \texttt{pip list}}

\item{\texttt{freeze}}
    \subitem{List all installed packages in requirements file format (see ``pip requirements'' below).}
    \subitem{Usage: \texttt{pip freeze}; \texttt{pip freeze | grep Flask}}

\end{itemize}

\subsection{pip requirements}
[TODO]

See \texttt{backend/requirements.txt} of Angular Momentum for an example.

\subsection{virtualenv}
[TODO]

\section{bundle}
[TODO]

\section{rake}
[TODO]

\section{guard}
[TODO]

\section{npm} 
[TODO]

\section{vagrant}
[TODO]

\end{document} 
​
